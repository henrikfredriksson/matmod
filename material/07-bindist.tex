% Created 2017-12-13 Wed 10:16
% Intended LaTeX compiler: pdflatex
\documentclass[presentation]{beamer}
\usepackage[T1]{fontenc}
\usepackage{fixltx2e}
\usepackage{graphicx}
\usepackage{longtable}
\usepackage{float}
\usepackage{wrapfig}
\usepackage{rotating}
\usepackage[normalem]{ulem}
\usepackage{amsmath}
\usepackage{textcomp}
\usepackage{marvosym}
\usepackage{wasysym}
\usepackage{amssymb}
\usepackage{hyperref}
\tolerance=1000
\usepackage{minted}
\institute[]{Blekinge Tekniska H�gskola}
\usepackage{tikz}
\usepackage[swedish, english]{babel}
\usetheme{Boadilla}
\author{Henrik Fredriksson}
\date{\today}
\title{MA1477 Matematisk modellering \newline Binomialf�rdelningen}
\hypersetup{
 pdfauthor={Henrik Fredriksson},
 pdftitle={MA1477 Matematisk modellering \newline Binomialf�rdelningen},
 pdfkeywords={},
 pdfsubject={},
 pdfcreator={Emacs 25.3.1 (Org mode 9.1.3)}, 
 pdflang={English}}
\begin{document}

\maketitle


\begin{frame}[fragile,label={sec:orgfd70d4b}]{Tv�punktsf�rdelad slumpvariabel}
\begin{itemize}
\item Vi ska unders�ka en stor grupp m�nniskor som har en saknar en viss
\end{itemize}
egenskap.

\begin{itemize}
\item L�t \(0 \leq \pi \leq\) vara andelen individer i en population som har en viss
egenskap.

\item \(\pi = 0\) om en slumpm�ssigt vald individ ur populationen saknar
egenskapen och \(\pi = 1\) om individen har egenskapen.
\end{itemize}

Det v�rde som variablen \(X\) som en slumpm�ssigt vald individ tilldelas
�r en \emph{tv�punktsf�rdelad slumpvariabel} med sannolikhetsf�rdelning


\begin{center}
\begin{tabular}{lrr}
\(x\) & 0 & 1\\
\hline
\(p(x)\) & \(1-\pi\) & \(\pi\)\\
\end{tabular}
\end{center}
\end{frame}


\begin{frame}[fragile,label={sec:org97be500}]{V�ntev�rde och varians}
V�ntev�rdet f�r \(X\) blir
\[
E(X) = \mu = \sum x\cdot p(x) = 0\cdot (1-\pi) + 1\cdot \pi = \pi
\]
Vi har �ven att 
\[
E(X^2) = \sum x^2 \cdot p(x) = \pi
\]
Det ger variansen
\[
\text{Var}(X) = E(X^2) - (E(X))^2 = \pi - \pi^2 = \pi(1-\pi)
\]
\end{frame}


\begin{frame}[fragile,label={sec:orgce1e399}]{Binomialf�rdelning}
Antag att vi g�r ett slumpn�ssigt urval om \(n\) individer fr�n stor
population och r�knar antalet individer som har en viss egenskap

Sannolikhetf�rdelningen blir
\[
\Pr(X=x) = \frac{n!}{x!(n-x)!}\cdot \pi^x \cdot (1-\pi)^{n-x}
\] 

d�r \(n! = n\cdot (n-1)(n-2)\cdot \ldots 2\cdot 1\).

\begin{itemize}
\item Sannolikhetsmodellen kallas \emph{binomialf�rdelningen} d�r \(n\) och \(\pi\) �r parametrar.
\end{itemize}
\end{frame}

\begin{frame}[fragile,label={sec:org3008e3e}]{Kodexempel (Exempel 6 sidan 87)}
 \begin{itemize}
\item Sannolikhet att en kund tackar "ja" till ett erbjudande �r \(60\%\)
\item Tjugo kunder tillfr�gas
\item Vad �r sannolikheten att h�gst tolv tackar "ja"
\end{itemize}

Vi ber�knar \(Pr(X\leq 12) = pr(0) + pr(1) + \ldots + pr(12)\) d�r \(X\) �r Bi(\(n = 20, \pi = 0.6\))

\begin{minted}[]{python}
from scipy.stats import binom
n = 20
p = 0.6

pr = 0
for i in range(13):
    pr += binom.pmf(i,n,p)

print(pr)
\end{minted}

\begin{verbatim}
0.584107062442
\end{verbatim}
\end{frame}

\begin{frame}[fragile,label={sec:org676274f}]{Kodexempel (Exempel 7 sidan 88)}
 \begin{minted}[]{python}
from scipy.stats import binom
n = 16
p = 0.3

pr = 0
for i in range(9):
    pr += binom.pmf(i,n,p)

for i in range(2):
    pr -= binom.pmf(i,n,p)


print(pr)
\end{minted}

\begin{verbatim}
0.94821488164
\end{verbatim}
\end{frame}
\end{document}